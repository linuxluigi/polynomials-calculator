% Generated by Sphinx.
\def\sphinxdocclass{report}
\newif\ifsphinxKeepOldNames \sphinxKeepOldNamestrue
\documentclass[letterpaper,10pt,ngerman]{sphinxmanual}
\usepackage{iftex}

\ifPDFTeX
  \usepackage[utf8]{inputenc}
\fi
\ifdefined\DeclareUnicodeCharacter
  \DeclareUnicodeCharacter{00A0}{\nobreakspace}
\fi
\usepackage{cmap}
\usepackage[T1]{fontenc}
\usepackage{amsmath,amssymb,amstext}
\usepackage{babel}
\usepackage{times}
\usepackage[Sonny]{fncychap}
\usepackage{longtable}
\usepackage{sphinx}
\usepackage{multirow}
\usepackage{eqparbox}


\addto\captionsngerman{\renewcommand{\figurename}{Abb.\@ }}
\addto\captionsngerman{\renewcommand{\tablename}{Tab.\@ }}
\SetupFloatingEnvironment{literal-block}{name=Quellcode }

\addto\extrasngerman{\def\pageautorefname{page}}

\setcounter{tocdepth}{1}


\title{Polynomials Calculator Dokumentation, Programmieren 2, Beleg 1, S0556166}
\date{Nov. 10, 2016}
\release{1.0.0}
\author{Steffen Exler}
\newcommand{\sphinxlogo}{\sphinxincludegraphics{logo.png}\par}
\renewcommand{\releasename}{Release}
\makeindex

\makeatletter
\def\PYG@reset{\let\PYG@it=\relax \let\PYG@bf=\relax%
    \let\PYG@ul=\relax \let\PYG@tc=\relax%
    \let\PYG@bc=\relax \let\PYG@ff=\relax}
\def\PYG@tok#1{\csname PYG@tok@#1\endcsname}
\def\PYG@toks#1+{\ifx\relax#1\empty\else%
    \PYG@tok{#1}\expandafter\PYG@toks\fi}
\def\PYG@do#1{\PYG@bc{\PYG@tc{\PYG@ul{%
    \PYG@it{\PYG@bf{\PYG@ff{#1}}}}}}}
\def\PYG#1#2{\PYG@reset\PYG@toks#1+\relax+\PYG@do{#2}}

\expandafter\def\csname PYG@tok@cs\endcsname{\def\PYG@tc##1{\textcolor[rgb]{0.25,0.50,0.56}{##1}}\def\PYG@bc##1{\setlength{\fboxsep}{0pt}\colorbox[rgb]{1.00,0.94,0.94}{\strut ##1}}}
\expandafter\def\csname PYG@tok@vi\endcsname{\def\PYG@tc##1{\textcolor[rgb]{0.73,0.38,0.84}{##1}}}
\expandafter\def\csname PYG@tok@gr\endcsname{\def\PYG@tc##1{\textcolor[rgb]{1.00,0.00,0.00}{##1}}}
\expandafter\def\csname PYG@tok@s\endcsname{\def\PYG@tc##1{\textcolor[rgb]{0.25,0.44,0.63}{##1}}}
\expandafter\def\csname PYG@tok@err\endcsname{\def\PYG@bc##1{\setlength{\fboxsep}{0pt}\fcolorbox[rgb]{1.00,0.00,0.00}{1,1,1}{\strut ##1}}}
\expandafter\def\csname PYG@tok@go\endcsname{\def\PYG@tc##1{\textcolor[rgb]{0.20,0.20,0.20}{##1}}}
\expandafter\def\csname PYG@tok@kr\endcsname{\let\PYG@bf=\textbf\def\PYG@tc##1{\textcolor[rgb]{0.00,0.44,0.13}{##1}}}
\expandafter\def\csname PYG@tok@mh\endcsname{\def\PYG@tc##1{\textcolor[rgb]{0.13,0.50,0.31}{##1}}}
\expandafter\def\csname PYG@tok@nn\endcsname{\let\PYG@bf=\textbf\def\PYG@tc##1{\textcolor[rgb]{0.05,0.52,0.71}{##1}}}
\expandafter\def\csname PYG@tok@o\endcsname{\def\PYG@tc##1{\textcolor[rgb]{0.40,0.40,0.40}{##1}}}
\expandafter\def\csname PYG@tok@nl\endcsname{\let\PYG@bf=\textbf\def\PYG@tc##1{\textcolor[rgb]{0.00,0.13,0.44}{##1}}}
\expandafter\def\csname PYG@tok@se\endcsname{\let\PYG@bf=\textbf\def\PYG@tc##1{\textcolor[rgb]{0.25,0.44,0.63}{##1}}}
\expandafter\def\csname PYG@tok@nd\endcsname{\let\PYG@bf=\textbf\def\PYG@tc##1{\textcolor[rgb]{0.33,0.33,0.33}{##1}}}
\expandafter\def\csname PYG@tok@nc\endcsname{\let\PYG@bf=\textbf\def\PYG@tc##1{\textcolor[rgb]{0.05,0.52,0.71}{##1}}}
\expandafter\def\csname PYG@tok@gp\endcsname{\let\PYG@bf=\textbf\def\PYG@tc##1{\textcolor[rgb]{0.78,0.36,0.04}{##1}}}
\expandafter\def\csname PYG@tok@mb\endcsname{\def\PYG@tc##1{\textcolor[rgb]{0.13,0.50,0.31}{##1}}}
\expandafter\def\csname PYG@tok@gd\endcsname{\def\PYG@tc##1{\textcolor[rgb]{0.63,0.00,0.00}{##1}}}
\expandafter\def\csname PYG@tok@ge\endcsname{\let\PYG@it=\textit}
\expandafter\def\csname PYG@tok@na\endcsname{\def\PYG@tc##1{\textcolor[rgb]{0.25,0.44,0.63}{##1}}}
\expandafter\def\csname PYG@tok@sh\endcsname{\def\PYG@tc##1{\textcolor[rgb]{0.25,0.44,0.63}{##1}}}
\expandafter\def\csname PYG@tok@s1\endcsname{\def\PYG@tc##1{\textcolor[rgb]{0.25,0.44,0.63}{##1}}}
\expandafter\def\csname PYG@tok@ne\endcsname{\def\PYG@tc##1{\textcolor[rgb]{0.00,0.44,0.13}{##1}}}
\expandafter\def\csname PYG@tok@ow\endcsname{\let\PYG@bf=\textbf\def\PYG@tc##1{\textcolor[rgb]{0.00,0.44,0.13}{##1}}}
\expandafter\def\csname PYG@tok@ss\endcsname{\def\PYG@tc##1{\textcolor[rgb]{0.32,0.47,0.09}{##1}}}
\expandafter\def\csname PYG@tok@gh\endcsname{\let\PYG@bf=\textbf\def\PYG@tc##1{\textcolor[rgb]{0.00,0.00,0.50}{##1}}}
\expandafter\def\csname PYG@tok@kp\endcsname{\def\PYG@tc##1{\textcolor[rgb]{0.00,0.44,0.13}{##1}}}
\expandafter\def\csname PYG@tok@ni\endcsname{\let\PYG@bf=\textbf\def\PYG@tc##1{\textcolor[rgb]{0.84,0.33,0.22}{##1}}}
\expandafter\def\csname PYG@tok@sr\endcsname{\def\PYG@tc##1{\textcolor[rgb]{0.14,0.33,0.53}{##1}}}
\expandafter\def\csname PYG@tok@nt\endcsname{\let\PYG@bf=\textbf\def\PYG@tc##1{\textcolor[rgb]{0.02,0.16,0.45}{##1}}}
\expandafter\def\csname PYG@tok@gu\endcsname{\let\PYG@bf=\textbf\def\PYG@tc##1{\textcolor[rgb]{0.50,0.00,0.50}{##1}}}
\expandafter\def\csname PYG@tok@nb\endcsname{\def\PYG@tc##1{\textcolor[rgb]{0.00,0.44,0.13}{##1}}}
\expandafter\def\csname PYG@tok@kt\endcsname{\def\PYG@tc##1{\textcolor[rgb]{0.56,0.13,0.00}{##1}}}
\expandafter\def\csname PYG@tok@mf\endcsname{\def\PYG@tc##1{\textcolor[rgb]{0.13,0.50,0.31}{##1}}}
\expandafter\def\csname PYG@tok@kd\endcsname{\let\PYG@bf=\textbf\def\PYG@tc##1{\textcolor[rgb]{0.00,0.44,0.13}{##1}}}
\expandafter\def\csname PYG@tok@vc\endcsname{\def\PYG@tc##1{\textcolor[rgb]{0.73,0.38,0.84}{##1}}}
\expandafter\def\csname PYG@tok@sb\endcsname{\def\PYG@tc##1{\textcolor[rgb]{0.25,0.44,0.63}{##1}}}
\expandafter\def\csname PYG@tok@mi\endcsname{\def\PYG@tc##1{\textcolor[rgb]{0.13,0.50,0.31}{##1}}}
\expandafter\def\csname PYG@tok@no\endcsname{\def\PYG@tc##1{\textcolor[rgb]{0.38,0.68,0.84}{##1}}}
\expandafter\def\csname PYG@tok@s2\endcsname{\def\PYG@tc##1{\textcolor[rgb]{0.25,0.44,0.63}{##1}}}
\expandafter\def\csname PYG@tok@kc\endcsname{\let\PYG@bf=\textbf\def\PYG@tc##1{\textcolor[rgb]{0.00,0.44,0.13}{##1}}}
\expandafter\def\csname PYG@tok@ch\endcsname{\let\PYG@it=\textit\def\PYG@tc##1{\textcolor[rgb]{0.25,0.50,0.56}{##1}}}
\expandafter\def\csname PYG@tok@cm\endcsname{\let\PYG@it=\textit\def\PYG@tc##1{\textcolor[rgb]{0.25,0.50,0.56}{##1}}}
\expandafter\def\csname PYG@tok@mo\endcsname{\def\PYG@tc##1{\textcolor[rgb]{0.13,0.50,0.31}{##1}}}
\expandafter\def\csname PYG@tok@gs\endcsname{\let\PYG@bf=\textbf}
\expandafter\def\csname PYG@tok@kn\endcsname{\let\PYG@bf=\textbf\def\PYG@tc##1{\textcolor[rgb]{0.00,0.44,0.13}{##1}}}
\expandafter\def\csname PYG@tok@nf\endcsname{\def\PYG@tc##1{\textcolor[rgb]{0.02,0.16,0.49}{##1}}}
\expandafter\def\csname PYG@tok@c1\endcsname{\let\PYG@it=\textit\def\PYG@tc##1{\textcolor[rgb]{0.25,0.50,0.56}{##1}}}
\expandafter\def\csname PYG@tok@k\endcsname{\let\PYG@bf=\textbf\def\PYG@tc##1{\textcolor[rgb]{0.00,0.44,0.13}{##1}}}
\expandafter\def\csname PYG@tok@bp\endcsname{\def\PYG@tc##1{\textcolor[rgb]{0.00,0.44,0.13}{##1}}}
\expandafter\def\csname PYG@tok@w\endcsname{\def\PYG@tc##1{\textcolor[rgb]{0.73,0.73,0.73}{##1}}}
\expandafter\def\csname PYG@tok@sd\endcsname{\let\PYG@it=\textit\def\PYG@tc##1{\textcolor[rgb]{0.25,0.44,0.63}{##1}}}
\expandafter\def\csname PYG@tok@sc\endcsname{\def\PYG@tc##1{\textcolor[rgb]{0.25,0.44,0.63}{##1}}}
\expandafter\def\csname PYG@tok@vg\endcsname{\def\PYG@tc##1{\textcolor[rgb]{0.73,0.38,0.84}{##1}}}
\expandafter\def\csname PYG@tok@il\endcsname{\def\PYG@tc##1{\textcolor[rgb]{0.13,0.50,0.31}{##1}}}
\expandafter\def\csname PYG@tok@cpf\endcsname{\let\PYG@it=\textit\def\PYG@tc##1{\textcolor[rgb]{0.25,0.50,0.56}{##1}}}
\expandafter\def\csname PYG@tok@cp\endcsname{\def\PYG@tc##1{\textcolor[rgb]{0.00,0.44,0.13}{##1}}}
\expandafter\def\csname PYG@tok@si\endcsname{\let\PYG@it=\textit\def\PYG@tc##1{\textcolor[rgb]{0.44,0.63,0.82}{##1}}}
\expandafter\def\csname PYG@tok@gi\endcsname{\def\PYG@tc##1{\textcolor[rgb]{0.00,0.63,0.00}{##1}}}
\expandafter\def\csname PYG@tok@nv\endcsname{\def\PYG@tc##1{\textcolor[rgb]{0.73,0.38,0.84}{##1}}}
\expandafter\def\csname PYG@tok@gt\endcsname{\def\PYG@tc##1{\textcolor[rgb]{0.00,0.27,0.87}{##1}}}
\expandafter\def\csname PYG@tok@m\endcsname{\def\PYG@tc##1{\textcolor[rgb]{0.13,0.50,0.31}{##1}}}
\expandafter\def\csname PYG@tok@c\endcsname{\let\PYG@it=\textit\def\PYG@tc##1{\textcolor[rgb]{0.25,0.50,0.56}{##1}}}
\expandafter\def\csname PYG@tok@sx\endcsname{\def\PYG@tc##1{\textcolor[rgb]{0.78,0.36,0.04}{##1}}}

\def\PYGZbs{\char`\\}
\def\PYGZus{\char`\_}
\def\PYGZob{\char`\{}
\def\PYGZcb{\char`\}}
\def\PYGZca{\char`\^}
\def\PYGZam{\char`\&}
\def\PYGZlt{\char`\<}
\def\PYGZgt{\char`\>}
\def\PYGZsh{\char`\#}
\def\PYGZpc{\char`\%}
\def\PYGZdl{\char`\$}
\def\PYGZhy{\char`\-}
\def\PYGZsq{\char`\'}
\def\PYGZdq{\char`\"}
\def\PYGZti{\char`\~}
% for compatibility with earlier versions
\def\PYGZat{@}
\def\PYGZlb{[}
\def\PYGZrb{]}
\makeatother

\renewcommand\PYGZsq{\textquotesingle}

\begin{document}
\shorthandoff{"}
\maketitle
\tableofcontents
\phantomsection\label{index::doc}



\chapter{Einleitung}
\label{index:polynomials-calculator-doc}\label{index:einleitung}

\section{Über das Programm}
\label{intro:uber-das-programm}\label{intro::doc}
Das Programm ``Polynomials Calculator'' ist ein reines Konsolenprogramm, welches dazu dient Polynome
bis zum n Grad zu Addieren, Subtrahieren, Multiplizieren und Dividieren.

Es wird dem User ermöglicht Polynome in einer Json Datei zu sichern und zu laden um zu einen späteren
Zeitpunkt weiter damit zu arbeiten.

Nach dem starten wird der Wizard-Modus gestartet um Polynome aus einer Json Datei zu laden und anschließend
wird das Menü gestartet, welche sich so lange wiederholt bis der User das Programm über das Menü schließt.

In Menü kann der User Polynome hinzufügen, bearbeiten und löschen aber auch Mathematische Operationen
Addieren, Subtrahieren, Multiplizieren und Dividieren ausführen sowie die Polynome als Json sichern
oder neu von der Json einlesen.

Das Projekt wurde mit JUnit 4 tests getestete und die test Klassen befinden sich bei den Quellcode dabei.
\begin{itemize}
\item {} 
Quellcode: \url{https://github.com/linuxluigi/polynomials-calculator}

\item {} 
Online Dokumentation: \url{http://polynomials-calculator.readthedocs.io/de/latest/}

\end{itemize}

\index{Git}
\index{Quellcode}
\index{Online Dokumentation}
\noindent\sphinxincludegraphics{{PolyCalc-Intro-1}.png}


\section{Kompilieren}
\label{intro:kompilieren}
\index{Kompilieren}
\index{Build}
\index{Build Artifacts}
\index{Maven}
Das Projekt wurde via Maven 2 konstruiert und kann mit ein Konsolen Befehl in einer Jar Datei Kompiliert werden, dafür
muss aber zuerst Maven 2 installiert werden, unter Ubuntu / Debian muss folgendes eingeben werden.

\begin{Verbatim}[commandchars=\\\{\}]
\PYGZdl{} sudo apt\PYGZhy{}get install maven2
\end{Verbatim}

Jetzt wurde Maven 2 installiert und nun kann das Projekt die abhänigkeiten installiert werden, test ausgeführt und
zur einer ausführbaren Jar ausgeben.

\begin{Verbatim}[commandchars=\\\{\}]
\PYGZdl{} mvn clean install
\end{Verbatim}

\noindent\sphinxincludegraphics{{PolyCalc-Compile-1}.png}


\section{Abhänigkeiten}
\label{intro:abhanigkeiten}
\index{Abhänigkeiten}
Das Projekt wurde als Maven 2 Modul geschrieben und verwendet folgende Maven Module.

Maven Projekt Website: \url{https://maven.apache.org/}


\subsection{Google GSON}
\label{intro:google-gson}
\index{Google GSON}
Gson ist eine Java Bibliothek die es ermöglicht Klassen und Variablen als Json Datei aus zu geben oder
ein String als Klasse oder Variable zu konvertieren.

\begin{DUlineblock}{0em}
\item[] Name: google-gson
\item[] Hersteller: Google Inc.
\item[] Version: 2.7
\item[] Link: \url{https://github.com/google/gson}
\end{DUlineblock}


\subsection{JUnit}
\label{intro:junit}
\index{JUnit}
Junit ist ein unit testing Framework für Java von Erich Gamma und Kent Beck.

\begin{DUlineblock}{0em}
\item[] Name: JUnit
\item[] Hersteller: Erich Gamma und Kent Beck
\item[] Version: 4.12
\item[] Link: \url{http://junit.org/junit4/}
\end{DUlineblock}


\chapter{Bedienung}
\label{index:bedienung}

\section{Wizard Modus}
\label{wizard:wizard-modus}\label{wizard::doc}

\subsection{Bedienung}
\label{wizard:bedienung}
Der Wizard Modus wird nur am Start des Programmes ausgeführt und dient zum Initialesieren
des Polynoms Array, Json Datei zu bestimmen und bei bedarf Polynome aus dieser Datei zu laden
und zur späteren Verwendung auf zu bereiten.

Nach dem start wird als erstes der Name und die Version des Programmes angezegt.
Anschließend wird abgefragt welche Json Datei zum sichern und laden der Polynome verwendet
werden soll und anschließend ob diese Datei geladen werden wird.

Die Aussagen die in den Eckigen Klammern stehen sind die Standartwerte, die verwendet werden sobald der
User nur Enter drückt, ohne eine weitere Eingabe zu tätigen.

\noindent\sphinxincludegraphics{{PolyCalc-Wizard-1}.png}

Die eingaben in dem Bild wird mit grüner Schrift dargestellt. Die Eingabe bedeutet in diesem Beispiel das
die Json Datei \sphinxtitleref{save.json} verwendet werden soll um die Polynome zu sichern und mit dem folgenden \sphinxtitleref{Y} lädt das
Programm die Polynome die in \sphinxtitleref{save.json} hinterlegt sind.

In diesem Beispiel wäre es kein unterschied ob der User 2 mal einfach nur Enter gedrückt hatte oder eine
vollstänge Eingabe getätigt hatte, da die Standart werde eingeben wurden.


\subsection{Datei Laden und Sichern}
\label{wizard:datei-laden-und-sichern}
\index{Json Laden und Sichern}
Bei dem Laden von der Json Datei, wird überprüft ob die Datei vorliegt, falls nicht wird sie neu erzeugt und
es wird ein Leeres Polynom Array zurückgeben. Falls die Datei exiestiert wird versucht den Inhalt, sobald
einer vorhanden ist, als Polynom Array zu konvertieren.

Ein Json Beispiel für 5 Polynome.

\begin{Verbatim}[commandchars=\\\{\}]
\PYG{p}{[}
  \PYG{p}{\PYGZob{}}\PYG{n+nt}{\PYGZdq{}polylist\PYGZdq{}}\PYG{p}{:}\PYG{p}{[}\PYG{l+m+mf}{44.3}\PYG{p}{,}\PYG{l+m+mf}{122345.0}\PYG{p}{,}\PYG{l+m+mf}{\PYGZhy{}5.654}\PYG{p}{,}\PYG{l+m+mf}{54.0}\PYG{p}{,}\PYG{l+m+mf}{416.0}\PYG{p}{,}\PYG{l+m+mf}{45.0}\PYG{p}{]}\PYG{p}{\PYGZcb{}}\PYG{p}{,}
  \PYG{p}{\PYGZob{}}\PYG{n+nt}{\PYGZdq{}polylist\PYGZdq{}}\PYG{p}{:}\PYG{p}{[}\PYG{l+m+mf}{5.0}\PYG{p}{,}\PYG{l+m+mf}{\PYGZhy{}16.0}\PYG{p}{,}\PYG{l+m+mf}{0.0}\PYG{p}{,}\PYG{l+m+mf}{\PYGZhy{}9.0}\PYG{p}{,}\PYG{l+m+mf}{10.0}\PYG{p}{,}\PYG{l+m+mf}{4.0}\PYG{p}{]}\PYG{p}{\PYGZcb{}}\PYG{p}{,}
  \PYG{p}{\PYGZob{}}\PYG{n+nt}{\PYGZdq{}polylist\PYGZdq{}}\PYG{p}{:}\PYG{p}{[}\PYG{l+m+mf}{0.0}\PYG{p}{,}\PYG{l+m+mf}{6.0}\PYG{p}{,}\PYG{l+m+mf}{2.0}\PYG{p}{]}\PYG{p}{\PYGZcb{}}\PYG{p}{,}
  \PYG{p}{\PYGZob{}}\PYG{n+nt}{\PYGZdq{}polylist\PYGZdq{}}\PYG{p}{:}\PYG{p}{[}\PYG{l+m+mf}{0.0}\PYG{p}{,}\PYG{l+m+mf}{6.0}\PYG{p}{,}\PYG{l+m+mf}{2.0}\PYG{p}{,}\PYG{l+m+mf}{3.0}\PYG{p}{]}\PYG{p}{\PYGZcb{}}\PYG{p}{,}
  \PYG{p}{\PYGZob{}}\PYG{n+nt}{\PYGZdq{}polylist\PYGZdq{}}\PYG{p}{:}\PYG{p}{[}\PYG{l+m+mf}{123.324}\PYG{p}{,}\PYG{l+m+mf}{123.0}\PYG{p}{,}\PYG{l+m+mf}{56.0}\PYG{p}{,}\PYG{l+m+mf}{\PYGZhy{}5612.42332}\PYG{p}{,}\PYG{l+m+mf}{654.234}\PYG{p}{,}\PYG{l+m+mf}{5.0}\PYG{p}{]}\PYG{p}{\PYGZcb{}}
\PYG{p}{]}
\end{Verbatim}


\section{Hauptmenü}
\label{menu:hauptmenu}\label{menu::doc}

\subsection{Inhalt}
\label{menu:inhalt}
Das Hauptmenu besitzt 12 Optionen mit folgendem Inhalten.

\noindent\begin{tabulary}{\linewidth}{|L|L|}
\hline
\textsf{\relax 
Option
\unskip}\relax &\textsf{\relax 
Inhalt
\unskip}\relax \\
\hline
1
&
Alle Polynome anzeigen die im Polynomarray hinterlegt sind.
\\
\hline
2  -  4
&
Polynome hinzufügen, bearbeiten oder löschen.
\\
\hline
5  -  9
&
Rechenoperationen mit Polynomen ausführen
\\
\hline
10 - 11
&
Json Datei Laden und Sichern
\\
\hline
12
&
Programm schließen
\\
\hline\end{tabulary}


\begin{DUlineblock}{0em}
\item[] 
\end{DUlineblock}

\noindent\sphinxincludegraphics{{PolyCalc-Menu-1}.png}


\subsection{Option 1: Alle Polynome anzeigen lassen}
\label{menu:option-1-alle-polynome-anzeigen-lassen}
\index{Polynome anzeigen}
Zeigt alle Polynome die dem Programm aktuell zu verfügung stehen. Die Polynome
werden untereinander aufgelistet und leicht lesbar Dargestellt. Die Zahl in der Eckigen Klammer {[}N{]}
dient der Übersicht, wieviele Polynome zu verfügung stehen.

\noindent\sphinxincludegraphics{{PolyCalc-Menu-2}.png}


\subsection{Option 2: Polynom hinzufügen}
\label{menu:option-2-polynom-hinzufugen}
\index{Polynom hinzufügen}
Als erstes wird aufgefordert die Länge des Polynomes ein zu geben. Die Länge heist in diesen Fall wie viele
Elemente das Polynom besitzt. Elemente mit den Wert 0 werden mitgezählt!

\begin{DUlineblock}{0em}
\item[] Länge = Größter Exponent + 1
\end{DUlineblock}

\begin{DUlineblock}{0em}
\item[] Beispiel:
\end{DUlineblock}
\begin{itemize}
\item {} 
\sphinxtitleref{f(x)=5+(-16x)+(-9x$^\text{3}$)+10x$^\text{4}$+4x$^\text{5}$} besitzt eine Länge von 6

\item {} 
\sphinxtitleref{f(x)=123.324+123x+56x$^\text{2}$+(-5612.423x$^\text{3}$)+654.234x$^\text{4}$+5x$^\text{5}$} besitzt eine Länge von 6

\item {} 
\sphinxtitleref{f(x)=+6x+2x$^\text{2}$+3x$^\text{3}$} besitzt eine Länge von 4

\end{itemize}

Nach der Eingabe der Länge wird aufgefordert jedes Element ein Wert aus dem Wertebereich der Reelen Zahlen
ein zu geben. Das Kommer für Reele Zahlen muss mit ein Punkt eingeben werden, ansonsten wird aufgeforert
das Element erneut ein zu geben.

\noindent{\hspace*{\fill}\sphinxincludegraphics[width=0.750\linewidth]{{PolyCalc-Menu-3}.png}\hspace*{\fill}}


\subsection{Option 3: Polynom bearbeiten}
\label{menu:option-3-polynom-bearbeiten}
\index{Polynom bearbeiten}
Um ein Polynom zu bearbeiten muss mindestens ein Polynom schon vorhanden sein. Sobald mindestens ein
Polynom vorhanden ist, erscheint die \sphinxtitleref{Polynom Liste}, die Zahl in der Eckigen Klammer {[}{]} vor jedes
Polynom ist der Wert der eingeben werden muss, um dieses Polynom zu bearbeiten.

Nachdem das Polynom ausgewählt wurde, wird aufgefordert jedes Element ein neuen Wert zu zu weisen,
der Wert in der Eckigen Klammer nach \sphinxtitleref{Element n {[}{]}} ist der aktuelle Wert des Element und durch drücken
der Enter Taste ohne weitere Werte ein zu geben bleibt der alte Wert unverändert.

Zum schluss wird das bearbeitete Polynom angezeigt.

\noindent{\hspace*{\fill}\sphinxincludegraphics[width=0.750\linewidth]{{PolyCalc-Menu-4}.png}\hspace*{\fill}}


\subsection{Option 3: Polynom löschen}
\label{menu:option-3-polynom-loschen}
\index{Polynom löschen}
Sobald Polynom löschen ausgewählt wurde, erscheint die \sphinxtitleref{Polynom Liste} woraus entschieden werden muss
welches Polynom gelöscht werden soll. Wenn die Zahl des Polynomes eingeben wurde, wird gefragt ob das
Polynom wirklich gelöscht werden soll, nur wenn `y' oder `Y' eingeben wurde, wird das Polynom wirlich
gelöscht.

\noindent{\hspace*{\fill}\sphinxincludegraphics[width=0.750\linewidth]{{PolyCalc-Menu-5}.png}\hspace*{\fill}}


\subsection{Option 5, 6 \& 8: Mathematische Operationen}
\label{menu:option-5-6-8-mathematische-operationen}
\index{Polynom Addition}
\index{Polynom Subtraktion}
\index{Polynom Multiplikation}
Das Eingabemuster bei Addition, Subtraktion und Multiplikation ist das gleiche.

Es erscheint die \sphinxtitleref{Polynom Liste} wo ausgewählt werden muss welches Polynom an erster Stelle Addiert,
Subtrahiert oder multipliziert werden soll und danach erscheint wieder die \sphinxtitleref{Polynom Liste} wo ausgewählt
welches Polynom an zweiter stelle der Optertion stehen soll.

Nach erfolgreicher eingabe wird die Mathematische Operation ausgeführt und das so neu erstandene Polynom
wird angezeigt und in der Polynom Liste automatisch gesichert.

\noindent\begin{tabulary}{\linewidth}{|L|L|L|}
\hline
\textsf{\relax 
Addtion
\unskip}\relax &\textsf{\relax 
Subtraktion
\unskip}\relax &\textsf{\relax 
Multiplikation
\unskip}\relax \\
\hline
\noindent\sphinxincludegraphics{{PolyCalc-Menu-6}.png}
&
\noindent\sphinxincludegraphics{{PolyCalc-Menu-7}.png}
&
\noindent\sphinxincludegraphics{{PolyCalc-Menu-8}.png}
\\
\hline\end{tabulary}



\subsection{Option 7: Polynom ableiten}
\label{menu:option-7-polynom-ableiten}
\index{Polynom ableiten}
\begin{DUlineblock}{0em}
\item[] In der erscheinenden \sphinxtitleref{Polynom Liste} das gewünschte Polynom auswählen und es erscheint das abgeleitete Polynom.
\end{DUlineblock}

\noindent{\hspace*{\fill}\sphinxincludegraphics[width=0.750\linewidth]{{PolyCalc-Menu-9}.png}\hspace*{\fill}}


\subsection{Option 9: Polynom Division}
\label{menu:option-9-polynom-division}
\index{Polynom Division}
\index{Horner Schema}
Die Polynom Division wird mit dem Hornerschema ausgeführt. Wie auch in anderen Polynom Mathematik Operationen muss
zuerst aus der \sphinxtitleref{Polynom Liste} das gewünschte Polynom ausgewählt werden und danach den Divisor.

Es wird nun das neue geteilelte Polynom ausgegeben und der Rest von der Division.

\noindent{\hspace*{\fill}\sphinxincludegraphics[width=0.750\linewidth]{{PolyCalc-Menu-10}.png}\hspace*{\fill}}


\subsection{Option 10: Json laden}
\label{menu:option-10-json-laden}
\index{Json laden}
Lädt die Polynome aus der Json Datei (festgelegt in Wizard am start des Programmes).

\noindent{\hspace*{\fill}\sphinxincludegraphics[width=0.750\linewidth]{{PolyCalc-Menu-11}.png}\hspace*{\fill}}


\subsection{Option 11: Json speichern}
\label{menu:option-11-json-speichern}
\index{Json sichern}
Sichert alle Polynome in die Json Datei, falls die Datei schon exiestiert wird sie mit den neuen Werten überschrieben.

\noindent{\hspace*{\fill}\sphinxincludegraphics[width=0.750\linewidth]{{PolyCalc-Menu-12}.png}\hspace*{\fill}}


\subsection{Option 12: Programm beenden}
\label{menu:option-12-programm-beenden}
\index{exit}
Beenden das Programm mit einer freundlichen Verabschiedung.

\noindent{\hspace*{\fill}\sphinxincludegraphics[width=0.750\linewidth]{{PolyCalc-Menu-13}.png}\hspace*{\fill}}


\chapter{Klassenbeschreibung}
\label{index:klassenbeschreibung}

\section{Polynomials Calculator}
\label{packages:polynomials-calculator}\label{packages::doc}

\subsection{com.linuxluigi.polynomial}
\label{com/linuxluigi/polynomial/package-index:package-com.linuxluigi.polynomial}\label{com/linuxluigi/polynomial/package-index:com-linuxluigi-polynomial}\label{com/linuxluigi/polynomial/package-index::doc}\index{com.linuxluigi.polynomial (package)}

\subsubsection{Main}
\label{com/linuxluigi/polynomial/Main:main}\label{com/linuxluigi/polynomial/Main::doc}\index{Main (Java class)}

\begin{fulllineitems}
\phantomsection\label{com/linuxluigi/polynomial/Main:com.linuxluigi.polynomial.Main}\pysigline{public class \sphinxbfcode{Main}}
Main Klasse, die das Terminal und PolynomialList initialesiert, außerdem staratet es den Wizard Modus der den User fragt ob die Json Datei geladen werden soll und deren Pfad definiert. Anschließend wird das Usermenu gestartet welches in Endloschleife arbeitet bis der User das Programm über das Menu beendet.
\begin{quote}\begin{description}
\item[{Author}] \leavevmode
Steffen Exler

\end{description}\end{quote}

\end{fulllineitems}



\paragraph{Methods}
\label{com/linuxluigi/polynomial/Main:methods}

\subparagraph{main}
\label{com/linuxluigi/polynomial/Main:id1}\index{main(String{[}{]}) (Java method)}

\begin{fulllineitems}
\phantomsection\label{com/linuxluigi/polynomial/Main:com.linuxluigi.polynomial.Main.main(String__)}\pysiglinewithargsret{public static void \sphinxbfcode{main}}{\href{http://docs.oracle.com/javase/6/docs/api/java/lang/String.html}{String}{[}{]}\emph{ args}}{}
Die Main Klasse zum starten des Userinterface, fragen nach der Json Datei Pfad und MainMenu in endlos Schleife starten äääöö
\begin{quote}\begin{description}
\item[{Parameter}] \leavevmode\begin{itemize}
\item {} 
\textbf{\texttt{args}} -- 
...


\end{itemize}

\end{description}\end{quote}

\end{fulllineitems}



\subsubsection{Polynomial}
\label{com/linuxluigi/polynomial/Polynomial:polynomial}\label{com/linuxluigi/polynomial/Polynomial::doc}\index{Polynomial (Java class)}

\begin{fulllineitems}
\phantomsection\label{com/linuxluigi/polynomial/Polynomial:com.linuxluigi.polynomial.Polynomial}\pysigline{public class \sphinxbfcode{Polynomial}}
Eine Klasse welche einzelne Polynome enthält die ausgegeben werden können, in einzelnen Elemente INT oder als Array. Gespeichert oder geändert werden kann das Objekt auch als Array oder über einzelne Elemente INT. Um auf einzelne Elemente INT zu zu greifen / ändern ist es möglich diese via die Funktionen get / set und ein Variable INT möglich.
\begin{itemize}
\item {} 
0 == x\textasciicircum{}0

\item {} 
1 == x\textasciicircum{}1

\item {} 
2 == x\textasciicircum{}2

\item {} 
3 == x\textasciicircum{}3

\item {} 
4 == x\textasciicircum{}4

\item {} 
5 == x\textasciicircum{}5

\end{itemize}

0 == Ergebnis, 1 == x\textasciicircum{}0, 7 == x\textasciicircum{}5
\begin{quote}\begin{description}
\item[{Author}] \leavevmode
Steffen Exler

\end{description}\end{quote}

\end{fulllineitems}



\paragraph{Constructors}
\label{com/linuxluigi/polynomial/Polynomial:constructors}

\subparagraph{Polynomial}
\label{com/linuxluigi/polynomial/Polynomial:id1}\index{Polynomial(double{[}{]}) (Java constructor)}

\begin{fulllineitems}
\phantomsection\label{com/linuxluigi/polynomial/Polynomial:com.linuxluigi.polynomial.Polynomial.Polynomial(double__)}\pysiglinewithargsret{public \sphinxbfcode{Polynomial}}{double{[}{]}\emph{ new\_polylist}}{}
Neuen Polynom aus ein vollständigen INT Array erzeugen
\begin{quote}\begin{description}
\item[{Parameter}] \leavevmode\begin{itemize}
\item {} 
\textbf{\texttt{new\_polylist}} -- Kompletter Polynom

\end{itemize}

\end{description}\end{quote}

\end{fulllineitems}



\subparagraph{Polynomial}
\label{com/linuxluigi/polynomial/Polynomial:id2}\index{Polynomial(int) (Java constructor)}

\begin{fulllineitems}
\phantomsection\label{com/linuxluigi/polynomial/Polynomial:com.linuxluigi.polynomial.Polynomial.Polynomial(int)}\pysiglinewithargsret{public \sphinxbfcode{Polynomial}}{int\emph{ length}}{}
Leeren Polynom mit der länge `length' erstellen.
\begin{quote}\begin{description}
\item[{Parameter}] \leavevmode\begin{itemize}
\item {} 
\textbf{\texttt{length}} -- Länge des Polynoms

\end{itemize}

\end{description}\end{quote}

\end{fulllineitems}



\paragraph{Methods}
\label{com/linuxluigi/polynomial/Polynomial:methods}

\subparagraph{Derivation}
\label{com/linuxluigi/polynomial/Polynomial:derivation}\index{Derivation() (Java method)}

\begin{fulllineitems}
\phantomsection\label{com/linuxluigi/polynomial/Polynomial:com.linuxluigi.polynomial.Polynomial.Derivation()}\pysiglinewithargsret{ \href{http://docs.oracle.com/javase/6/docs/api/java/lang/String.html}{String} \sphinxbfcode{Derivation}}{}{}
Gibt die 1. Ableitung des Polynomes zurück
\begin{quote}\begin{description}
\item[{Rückgabe}] \leavevmode
Menschlich lesbare 1. Ableitung des Polynomes

\end{description}\end{quote}

\end{fulllineitems}



\subparagraph{get}
\label{com/linuxluigi/polynomial/Polynomial:get}\index{get() (Java method)}

\begin{fulllineitems}
\phantomsection\label{com/linuxluigi/polynomial/Polynomial:com.linuxluigi.polynomial.Polynomial.get()}\pysiglinewithargsret{public double{[}{]} \sphinxbfcode{get}}{}{}
Gibt den Polynom als INT Array zurück
\begin{quote}\begin{description}
\item[{Rückgabe}] \leavevmode
Gibt komplettes Polynom zurück

\end{description}\end{quote}

\end{fulllineitems}



\subparagraph{get}
\label{com/linuxluigi/polynomial/Polynomial:id3}\index{get(int) (Java method)}

\begin{fulllineitems}
\phantomsection\label{com/linuxluigi/polynomial/Polynomial:com.linuxluigi.polynomial.Polynomial.get(int)}\pysiglinewithargsret{public double \sphinxbfcode{get}}{int\emph{ number}}{}
Gibt ein Element des Polynomes zurück
\begin{quote}\begin{description}
\item[{Parameter}] \leavevmode\begin{itemize}
\item {} 
\textbf{\texttt{number}} -- Element nummer des Polynomes this.polylist{[}number{]}

\end{itemize}

\item[{Rückgabe}] \leavevmode
Wert des Polynom Element

\end{description}\end{quote}

\end{fulllineitems}



\subparagraph{get\_as\_human\_readable}
\label{com/linuxluigi/polynomial/Polynomial:get-as-human-readable}\index{get\_as\_human\_readable() (Java method)}

\begin{fulllineitems}
\phantomsection\label{com/linuxluigi/polynomial/Polynomial:com.linuxluigi.polynomial.Polynomial.get_as_human_readable()}\pysiglinewithargsret{ \href{http://docs.oracle.com/javase/6/docs/api/java/lang/String.html}{String} \sphinxbfcode{get\_as\_human\_readable}}{}{}
Wandelt das Polynom Array als Menschlich lesbaren Polynom um
\begin{quote}\begin{description}
\item[{Rückgabe}] \leavevmode
Polynom als lesbaren String

\end{description}\end{quote}

\end{fulllineitems}



\subparagraph{length}
\label{com/linuxluigi/polynomial/Polynomial:length}\index{length() (Java method)}

\begin{fulllineitems}
\phantomsection\label{com/linuxluigi/polynomial/Polynomial:com.linuxluigi.polynomial.Polynomial.length()}\pysiglinewithargsret{public int \sphinxbfcode{length}}{}{}
Gibt die Länge des Polynomes zurück
\begin{quote}\begin{description}
\item[{Rückgabe}] \leavevmode
Int länge des Polynomes Array

\end{description}\end{quote}

\end{fulllineitems}



\subparagraph{set}
\label{com/linuxluigi/polynomial/Polynomial:set}\index{set(double{[}{]}) (Java method)}

\begin{fulllineitems}
\phantomsection\label{com/linuxluigi/polynomial/Polynomial:com.linuxluigi.polynomial.Polynomial.set(double__)}\pysiglinewithargsret{public void \sphinxbfcode{set}}{double{[}{]}\emph{ new\_polylist}}{}
Überschreibt den Polynom mit einem neuen `new\_polylist'
\begin{quote}\begin{description}
\item[{Parameter}] \leavevmode\begin{itemize}
\item {} 
\textbf{\texttt{new\_polylist}} -- Vollständiger Polynom als INT Array

\end{itemize}

\end{description}\end{quote}

\end{fulllineitems}



\subparagraph{set}
\label{com/linuxluigi/polynomial/Polynomial:id4}\index{set(int, double) (Java method)}

\begin{fulllineitems}
\phantomsection\label{com/linuxluigi/polynomial/Polynomial:com.linuxluigi.polynomial.Polynomial.set(int, double)}\pysiglinewithargsret{public void \sphinxbfcode{set}}{int\emph{ poly\_number}, double\emph{ poly\_value}}{}
Überschreibt ein Element des Polynomes
\begin{quote}\begin{description}
\item[{Parameter}] \leavevmode\begin{itemize}
\item {} 
\textbf{\texttt{poly\_number}} -- Element des Polynomes

\item {} 
\textbf{\texttt{poly\_value}} -- Wert des neuen Element im Polynom

\end{itemize}

\end{description}\end{quote}

\end{fulllineitems}



\subsubsection{PolynomialList}
\label{com/linuxluigi/polynomial/PolynomialList:polynomiallist}\label{com/linuxluigi/polynomial/PolynomialList::doc}\index{PolynomialList (Java class)}

\begin{fulllineitems}
\phantomsection\label{com/linuxluigi/polynomial/PolynomialList:com.linuxluigi.polynomial.PolynomialList}\pysigline{ class \sphinxbfcode{PolynomialList}}
Ein Polynom Klasse Array welche mitunter folgende funktionen mitbringt:
\begin{itemize}
\item {} 
Einzelne Polynome aus den Polynom{[}{]} ausgeben

\item {} 
Polynome miteinander multiplizieren, addieren und subtrahieren

\item {} 
Einzelne Polynome löschen, bearbeiten oder neu hinzufügen

\item {} 
Polynom{[}{]} bilden durch laden einer Json Datei

\item {} 
Die eigene Klasse als Json Datei speichern

\end{itemize}

\end{fulllineitems}



\paragraph{Constructors}
\label{com/linuxluigi/polynomial/PolynomialList:constructors}

\subparagraph{PolynomialList}
\label{com/linuxluigi/polynomial/PolynomialList:id1}\index{PolynomialList() (Java constructor)}

\begin{fulllineitems}
\phantomsection\label{com/linuxluigi/polynomial/PolynomialList:com.linuxluigi.polynomial.PolynomialList.PolynomialList()}\pysiglinewithargsret{public \sphinxbfcode{PolynomialList}}{}{}
Konstruktor Erstellt ein neues leeres Polynomial{[}{]}

\end{fulllineitems}



\paragraph{Methods}
\label{com/linuxluigi/polynomial/PolynomialList:methods}

\subparagraph{add}
\label{com/linuxluigi/polynomial/PolynomialList:add}\index{add(Polynomial) (Java method)}

\begin{fulllineitems}
\phantomsection\label{com/linuxluigi/polynomial/PolynomialList:com.linuxluigi.polynomial.PolynomialList.add(Polynomial)}\pysiglinewithargsret{public void \sphinxbfcode{add}}{{\hyperref[com/linuxluigi/polynomial/Polynomial:com.linuxluigi.polynomial.Polynomial]{\sphinxcrossref{Polynomial}}}\emph{ newPolynomial}}{}
Hängt ein neues Polynomial an Polynomial{[}{]} an
\begin{quote}\begin{description}
\item[{Parameter}] \leavevmode\begin{itemize}
\item {} 
\textbf{\texttt{newPolynomial}} -- neues Polynomial welches angehängt werden soll

\end{itemize}

\end{description}\end{quote}

\end{fulllineitems}



\subparagraph{delte}
\label{com/linuxluigi/polynomial/PolynomialList:delte}\index{delte(int) (Java method)}

\begin{fulllineitems}
\phantomsection\label{com/linuxluigi/polynomial/PolynomialList:com.linuxluigi.polynomial.PolynomialList.delte(int)}\pysiglinewithargsret{ void \sphinxbfcode{delte}}{int\emph{ PolynomialNumber}}{}
Löscht ein Element aus den Polynomial{[}{]}
\begin{quote}\begin{description}
\item[{Parameter}] \leavevmode\begin{itemize}
\item {} 
\textbf{\texttt{PolynomialNumber}} -- Element des Polynomial{[}{]} welches gelöscht werden soll

\end{itemize}

\end{description}\end{quote}

\end{fulllineitems}



\subparagraph{get\_FileName}
\label{com/linuxluigi/polynomial/PolynomialList:get-filename}\index{get\_FileName() (Java method)}

\begin{fulllineitems}
\phantomsection\label{com/linuxluigi/polynomial/PolynomialList:com.linuxluigi.polynomial.PolynomialList.get_FileName()}\pysiglinewithargsret{ \href{http://docs.oracle.com/javase/6/docs/api/java/lang/String.html}{String} \sphinxbfcode{get\_FileName}}{}{}
Gibt den Json Datei String zurück
\begin{quote}\begin{description}
\item[{Rückgabe}] \leavevmode
Json Datei namen als String

\end{description}\end{quote}

\end{fulllineitems}



\subparagraph{get\_PolylList}
\label{com/linuxluigi/polynomial/PolynomialList:get-polyllist}\index{get\_PolylList() (Java method)}

\begin{fulllineitems}
\phantomsection\label{com/linuxluigi/polynomial/PolynomialList:com.linuxluigi.polynomial.PolynomialList.get_PolylList()}\pysiglinewithargsret{ {\hyperref[com/linuxluigi/polynomial/Polynomial:com.linuxluigi.polynomial.Polynomial]{\sphinxcrossref{Polynomial}}}{[}{]} \sphinxbfcode{get\_PolylList}}{}{}
Gibt das Polynomial{[}{]} zurück
\begin{quote}\begin{description}
\item[{Rückgabe}] \leavevmode
Polynomial{[}{]}

\end{description}\end{quote}

\end{fulllineitems}



\subparagraph{get\_Polynomial}
\label{com/linuxluigi/polynomial/PolynomialList:get-polynomial}\index{get\_Polynomial(int) (Java method)}

\begin{fulllineitems}
\phantomsection\label{com/linuxluigi/polynomial/PolynomialList:com.linuxluigi.polynomial.PolynomialList.get_Polynomial(int)}\pysiglinewithargsret{ {\hyperref[com/linuxluigi/polynomial/Polynomial:com.linuxluigi.polynomial.Polynomial]{\sphinxcrossref{Polynomial}}} \sphinxbfcode{get\_Polynomial}}{int\emph{ PolynomialNumber}}{}
Gibt ein einzelnes Polynomial aus dem Polynomial{[}{]} zurück
\begin{quote}\begin{description}
\item[{Parameter}] \leavevmode\begin{itemize}
\item {} 
\textbf{\texttt{PolynomialNumber}} -- Element des Polynomial{[}{]} welches zurück gegeben werden soll

\end{itemize}

\item[{Rückgabe}] \leavevmode
Polynomial Objekt

\end{description}\end{quote}

\end{fulllineitems}



\subparagraph{length}
\label{com/linuxluigi/polynomial/PolynomialList:length}\index{length() (Java method)}

\begin{fulllineitems}
\phantomsection\label{com/linuxluigi/polynomial/PolynomialList:com.linuxluigi.polynomial.PolynomialList.length()}\pysiglinewithargsret{public int \sphinxbfcode{length}}{}{}
Gibt die länge des Polynomial{[}{]} zurück
\begin{quote}\begin{description}
\item[{Rückgabe}] \leavevmode
Int länge des Polynomial{[}{]}

\end{description}\end{quote}

\end{fulllineitems}



\subparagraph{load}
\label{com/linuxluigi/polynomial/PolynomialList:load}\index{load() (Java method)}

\begin{fulllineitems}
\phantomsection\label{com/linuxluigi/polynomial/PolynomialList:com.linuxluigi.polynomial.PolynomialList.load()}\pysiglinewithargsret{ void \sphinxbfcode{load}}{}{}
Ersetzt das vorhandene Polynomial{[}{]} mit der aus der this.file Json Datei angeben Werten Polynomial{[}{]}

\end{fulllineitems}



\subparagraph{mathAddSub}
\label{com/linuxluigi/polynomial/PolynomialList:mathaddsub}\index{mathAddSub(Polynomial, Polynomial, boolean) (Java method)}

\begin{fulllineitems}
\phantomsection\label{com/linuxluigi/polynomial/PolynomialList:com.linuxluigi.polynomial.PolynomialList.mathAddSub(Polynomial, Polynomial, boolean)}\pysiglinewithargsret{ {\hyperref[com/linuxluigi/polynomial/Polynomial:com.linuxluigi.polynomial.Polynomial]{\sphinxcrossref{Polynomial}}} \sphinxbfcode{mathAddSub}}{{\hyperref[com/linuxluigi/polynomial/Polynomial:com.linuxluigi.polynomial.Polynomial]{\sphinxcrossref{Polynomial}}}\emph{ Polynomial\_1}, {\hyperref[com/linuxluigi/polynomial/Polynomial:com.linuxluigi.polynomial.Polynomial]{\sphinxcrossref{Polynomial}}}\emph{ Polynomial\_2}, boolean\emph{ operator}}{}
Addiert oder Subtraiert 2 Polynome miteinander, gibt dieses als Polynomial Klasse zurück und fügt es in Polynomial{[}{]} hinzu
\begin{quote}\begin{description}
\item[{Parameter}] \leavevmode\begin{itemize}
\item {} 
\textbf{\texttt{Polynomial\_1}} -- Polynom 1 welche zu Polynom 2 addiert wird

\item {} 
\textbf{\texttt{Polynomial\_2}} -- Polynom 2 welche zu Polynom 1 addiert wird

\item {} 
\textbf{\texttt{operator}} -- 1 == +, 0 == -

\end{itemize}

\item[{Rückgabe}] \leavevmode
Neues Polynomial welches durch die Berechnung entstand

\end{description}\end{quote}

\end{fulllineitems}



\subparagraph{mathHorner}
\label{com/linuxluigi/polynomial/PolynomialList:mathhorner}\index{mathHorner(Polynomial, double) (Java method)}

\begin{fulllineitems}
\phantomsection\label{com/linuxluigi/polynomial/PolynomialList:com.linuxluigi.polynomial.PolynomialList.mathHorner(Polynomial, double)}\pysiglinewithargsret{ double \sphinxbfcode{mathHorner}}{{\hyperref[com/linuxluigi/polynomial/Polynomial:com.linuxluigi.polynomial.Polynomial]{\sphinxcrossref{Polynomial}}}\emph{ Polynomial}, double\emph{ divisor}}{}
Polynomdivision nach dem Horner Schema, bei erfogreicher Division wird das neue Polynom Polynomial{[}{]} angehängt
\begin{quote}\begin{description}
\item[{Parameter}] \leavevmode\begin{itemize}
\item {} 
\textbf{\texttt{Polynomial}} -- Polynom welches dividiert werden soll

\item {} 
\textbf{\texttt{divisor}} -- Die Zahl mit der das Polynom dividiert werden soll

\end{itemize}

\item[{Rückgabe}] \leavevmode
Rest in Double

\end{description}\end{quote}

\end{fulllineitems}



\subparagraph{mathMultiply}
\label{com/linuxluigi/polynomial/PolynomialList:mathmultiply}\index{mathMultiply(Polynomial, Polynomial) (Java method)}

\begin{fulllineitems}
\phantomsection\label{com/linuxluigi/polynomial/PolynomialList:com.linuxluigi.polynomial.PolynomialList.mathMultiply(Polynomial, Polynomial)}\pysiglinewithargsret{ {\hyperref[com/linuxluigi/polynomial/Polynomial:com.linuxluigi.polynomial.Polynomial]{\sphinxcrossref{Polynomial}}} \sphinxbfcode{mathMultiply}}{{\hyperref[com/linuxluigi/polynomial/Polynomial:com.linuxluigi.polynomial.Polynomial]{\sphinxcrossref{Polynomial}}}\emph{ Polynomial\_1}, {\hyperref[com/linuxluigi/polynomial/Polynomial:com.linuxluigi.polynomial.Polynomial]{\sphinxcrossref{Polynomial}}}\emph{ Polynomial\_2}}{}
Multipliziert 2 Polynome miteinander und speichert das Polynom in PolylList
\begin{quote}\begin{description}
\item[{Parameter}] \leavevmode\begin{itemize}
\item {} 
\textbf{\texttt{Polynomial\_1}} -- Polynom 1 welches zu Polynom 2 multipliziert werden soll

\item {} 
\textbf{\texttt{Polynomial\_2}} -- Polynom 2 welches zu Polynom 1 multipliziert werden soll

\end{itemize}

\item[{Rückgabe}] \leavevmode
neues multipliziertes Polynom

\end{description}\end{quote}

\end{fulllineitems}



\subparagraph{randomPolynomial}
\label{com/linuxluigi/polynomial/PolynomialList:randompolynomial}\index{randomPolynomial(int, boolean) (Java method)}

\begin{fulllineitems}
\phantomsection\label{com/linuxluigi/polynomial/PolynomialList:com.linuxluigi.polynomial.PolynomialList.randomPolynomial(int, boolean)}\pysiglinewithargsret{ {\hyperref[com/linuxluigi/polynomial/Polynomial:com.linuxluigi.polynomial.Polynomial]{\sphinxcrossref{Polynomial}}} \sphinxbfcode{randomPolynomial}}{int\emph{ length}, boolean\emph{ random}}{}
Erstellt ein Polynomial mit der Länge length und wenn random wahr ist, mit festen Werten
\begin{quote}\begin{description}
\item[{Parameter}] \leavevmode\begin{itemize}
\item {} 
\textbf{\texttt{length}} -- länge des Beispiel Polynomes

\item {} 
\textbf{\texttt{random}} -- Polynom bekommt feste Werte zugewiesen mit {[}i{]} = i

\end{itemize}

\item[{Rückgabe}] \leavevmode
zufälliges neues Polynomial

\end{description}\end{quote}

\end{fulllineitems}



\subparagraph{randomPolynomialArray}
\label{com/linuxluigi/polynomial/PolynomialList:randompolynomialarray}\index{randomPolynomialArray(int, int, boolean) (Java method)}

\begin{fulllineitems}
\phantomsection\label{com/linuxluigi/polynomial/PolynomialList:com.linuxluigi.polynomial.PolynomialList.randomPolynomialArray(int, int, boolean)}\pysiglinewithargsret{ {\hyperref[com/linuxluigi/polynomial/Polynomial:com.linuxluigi.polynomial.Polynomial]{\sphinxcrossref{Polynomial}}}{[}{]} \sphinxbfcode{randomPolynomialArray}}{int\emph{ arrayLength}, int\emph{ PolynomialLength}, boolean\emph{ random}}{}
Erstellt ein Polynomial{[}{]} mit zufalls Zahlen und arrayLength länge, die länge der Polynome wird mit PolynomialLength bestimmt
\begin{quote}\begin{description}
\item[{Parameter}] \leavevmode\begin{itemize}
\item {} 
\textbf{\texttt{arrayLength}} -- Länge von Polynomial{[}{]}

\item {} 
\textbf{\texttt{PolynomialLength}} -- Länge des Polynomial

\item {} 
\textbf{\texttt{random}} -- Polynom bekommt feste Werte zugewiesen mit {[}i{]} = i

\end{itemize}

\item[{Rückgabe}] \leavevmode
zufälliges neues Polynomial{[}{]}

\end{description}\end{quote}

\end{fulllineitems}



\subparagraph{save}
\label{com/linuxluigi/polynomial/PolynomialList:save}\index{save() (Java method)}

\begin{fulllineitems}
\phantomsection\label{com/linuxluigi/polynomial/PolynomialList:com.linuxluigi.polynomial.PolynomialList.save()}\pysiglinewithargsret{ void \sphinxbfcode{save}}{}{}
Speichert Polynomial{[}{]} in this.file angeben Datei als Json format ab

\end{fulllineitems}



\subparagraph{set}
\label{com/linuxluigi/polynomial/PolynomialList:set}\index{set(int, Polynomial) (Java method)}

\begin{fulllineitems}
\phantomsection\label{com/linuxluigi/polynomial/PolynomialList:com.linuxluigi.polynomial.PolynomialList.set(int, Polynomial)}\pysiglinewithargsret{public void \sphinxbfcode{set}}{int\emph{ ArrayNumber}, {\hyperref[com/linuxluigi/polynomial/Polynomial:com.linuxluigi.polynomial.Polynomial]{\sphinxcrossref{Polynomial}}}\emph{ newPolynomial}}{}
Überschreibt ein Polynomial aus Polynomial{[}{]} mit einen neuem Polynomial
\begin{quote}\begin{description}
\item[{Parameter}] \leavevmode\begin{itemize}
\item {} 
\textbf{\texttt{ArrayNumber}} -- Element nummer des zu überschreibenen Polynomial

\item {} 
\textbf{\texttt{newPolynomial}} -- Neues Polynomial welches das alte überschreiben soll

\end{itemize}

\end{description}\end{quote}

\end{fulllineitems}



\subparagraph{set\_file}
\label{com/linuxluigi/polynomial/PolynomialList:set-file}\index{set\_file(String) (Java method)}

\begin{fulllineitems}
\phantomsection\label{com/linuxluigi/polynomial/PolynomialList:com.linuxluigi.polynomial.PolynomialList.set_file(String)}\pysiglinewithargsret{ void \sphinxbfcode{set\_file}}{\href{http://docs.oracle.com/javase/6/docs/api/java/lang/String.html}{String}\emph{ FileName}}{}
Setzt den Namen und Pfad der Json Datei
\begin{quote}\begin{description}
\item[{Parameter}] \leavevmode\begin{itemize}
\item {} 
\textbf{\texttt{FileName}} -- Datei Namen und Pfad der neuen Json Datei

\end{itemize}

\end{description}\end{quote}

\end{fulllineitems}



\subsubsection{PolynomialListTest}
\label{com/linuxluigi/polynomial/PolynomialListTest:polynomiallisttest}\label{com/linuxluigi/polynomial/PolynomialListTest::doc}\index{PolynomialListTest (Java class)}

\begin{fulllineitems}
\phantomsection\label{com/linuxluigi/polynomial/PolynomialListTest:com.linuxluigi.polynomial.PolynomialListTest}\pysigline{public class \sphinxbfcode{PolynomialListTest}}
Created by Steffen Exler on 03.11.16.

\end{fulllineitems}



\paragraph{Methods}
\label{com/linuxluigi/polynomial/PolynomialListTest:methods}

\subparagraph{add}
\label{com/linuxluigi/polynomial/PolynomialListTest:add}\index{add() (Java method)}

\begin{fulllineitems}
\phantomsection\label{com/linuxluigi/polynomial/PolynomialListTest:com.linuxluigi.polynomial.PolynomialListTest.add()}\pysiglinewithargsret{public void \sphinxbfcode{add}}{}{}
Erstellt ein PolynomialList Objekt und füllt es mit zufallswerten und überprüft ob die Ausgabe mit der Eingabe übereinstimmt, außerdem werden noch Vordefinierte double{[}{]} Werte als Polynom erstellt, PolynomialList angehängt und überprüft ob hier auch die Eingabe und Ausgabe übereinstimmt.
\begin{quote}\begin{description}
\item[{Wirft}] \leavevmode\begin{itemize}
\item {} 
\href{http://docs.oracle.com/javase/6/docs/api/java/lang/Exception.html}{\textbf{\texttt{Exception}}} -- 

\end{itemize}

\end{description}\end{quote}

\end{fulllineitems}



\subparagraph{delte}
\label{com/linuxluigi/polynomial/PolynomialListTest:delte}\index{delte() (Java method)}

\begin{fulllineitems}
\phantomsection\label{com/linuxluigi/polynomial/PolynomialListTest:com.linuxluigi.polynomial.PolynomialListTest.delte()}\pysiglinewithargsret{public void \sphinxbfcode{delte}}{}{}
Erzeugt ein zufälliges PolynomialList und löscht zufällig einzelne Werte heraus Test dann ob die länge von PolynomialList -1 ist und überprüft ob das Polynom wirklich aus PolynomialList gelöscht wurde
\begin{quote}\begin{description}
\item[{Wirft}] \leavevmode\begin{itemize}
\item {} 
\href{http://docs.oracle.com/javase/6/docs/api/java/lang/Exception.html}{\textbf{\texttt{Exception}}} -- 

\end{itemize}

\end{description}\end{quote}

\end{fulllineitems}



\subparagraph{mathAddSub}
\label{com/linuxluigi/polynomial/PolynomialListTest:mathaddsub}\index{mathAddSub() (Java method)}

\begin{fulllineitems}
\phantomsection\label{com/linuxluigi/polynomial/PolynomialListTest:com.linuxluigi.polynomial.PolynomialListTest.mathAddSub()}\pysiglinewithargsret{public void \sphinxbfcode{mathAddSub}}{}{}
Test Addition und Subtraktion von Polynome mit zufallszahlen und fest Vordefinierten Zahlen
\begin{quote}\begin{description}
\item[{Wirft}] \leavevmode\begin{itemize}
\item {} 
\href{http://docs.oracle.com/javase/6/docs/api/java/lang/Exception.html}{\textbf{\texttt{Exception}}} -- 

\end{itemize}

\end{description}\end{quote}

\end{fulllineitems}



\subparagraph{mathHorner}
\label{com/linuxluigi/polynomial/PolynomialListTest:mathhorner}\index{mathHorner() (Java method)}

\begin{fulllineitems}
\phantomsection\label{com/linuxluigi/polynomial/PolynomialListTest:com.linuxluigi.polynomial.PolynomialListTest.mathHorner()}\pysiglinewithargsret{public void \sphinxbfcode{mathHorner}}{}{}
Test Hornerschema nach festen Werten
\begin{quote}\begin{description}
\item[{Wirft}] \leavevmode\begin{itemize}
\item {} 
\href{http://docs.oracle.com/javase/6/docs/api/java/lang/Exception.html}{\textbf{\texttt{Exception}}} -- 

\end{itemize}

\end{description}\end{quote}

\end{fulllineitems}



\subparagraph{mathMultiply}
\label{com/linuxluigi/polynomial/PolynomialListTest:mathmultiply}\index{mathMultiply() (Java method)}

\begin{fulllineitems}
\phantomsection\label{com/linuxluigi/polynomial/PolynomialListTest:com.linuxluigi.polynomial.PolynomialListTest.mathMultiply()}\pysiglinewithargsret{public void \sphinxbfcode{mathMultiply}}{}{}
Test Multiplikation von Polynome mit zufallszahlen und fest Vordefinierten Zahlen
\begin{quote}\begin{description}
\item[{Wirft}] \leavevmode\begin{itemize}
\item {} 
\href{http://docs.oracle.com/javase/6/docs/api/java/lang/Exception.html}{\textbf{\texttt{Exception}}} -- 

\end{itemize}

\end{description}\end{quote}

\end{fulllineitems}



\subsubsection{PolynomialTest}
\label{com/linuxluigi/polynomial/PolynomialTest:polynomialtest}\label{com/linuxluigi/polynomial/PolynomialTest::doc}\index{PolynomialTest (Java class)}

\begin{fulllineitems}
\phantomsection\label{com/linuxluigi/polynomial/PolynomialTest:com.linuxluigi.polynomial.PolynomialTest}\pysigline{public class \sphinxbfcode{PolynomialTest}}
Created by Steffen Exler on 01.11.16.

\end{fulllineitems}



\paragraph{Methods}
\label{com/linuxluigi/polynomial/PolynomialTest:methods}

\subparagraph{derivation}
\label{com/linuxluigi/polynomial/PolynomialTest:derivation}\index{derivation() (Java method)}

\begin{fulllineitems}
\phantomsection\label{com/linuxluigi/polynomial/PolynomialTest:com.linuxluigi.polynomial.PolynomialTest.derivation()}\pysiglinewithargsret{public void \sphinxbfcode{derivation}}{}{}
Erste Ableitung Test
\begin{quote}\begin{description}
\item[{Wirft}] \leavevmode\begin{itemize}
\item {} 
\href{http://docs.oracle.com/javase/6/docs/api/java/lang/Exception.html}{\textbf{\texttt{Exception}}} -- 

\end{itemize}

\end{description}\end{quote}

\end{fulllineitems}



\subparagraph{get}
\label{com/linuxluigi/polynomial/PolynomialTest:get}\index{get() (Java method)}

\begin{fulllineitems}
\phantomsection\label{com/linuxluigi/polynomial/PolynomialTest:com.linuxluigi.polynomial.PolynomialTest.get()}\pysiglinewithargsret{public void \sphinxbfcode{get}}{}{}
Testet beide get Varianten mit zufalls und festen Werten
\begin{quote}\begin{description}
\item[{Wirft}] \leavevmode\begin{itemize}
\item {} 
\href{http://docs.oracle.com/javase/6/docs/api/java/lang/Exception.html}{\textbf{\texttt{Exception}}} -- 

\end{itemize}

\end{description}\end{quote}

\end{fulllineitems}



\subparagraph{get\_as\_human\_readable}
\label{com/linuxluigi/polynomial/PolynomialTest:get-as-human-readable}\index{get\_as\_human\_readable() (Java method)}

\begin{fulllineitems}
\phantomsection\label{com/linuxluigi/polynomial/PolynomialTest:com.linuxluigi.polynomial.PolynomialTest.get_as_human_readable()}\pysiglinewithargsret{public void \sphinxbfcode{get\_as\_human\_readable}}{}{}
\end{fulllineitems}



\subparagraph{length}
\label{com/linuxluigi/polynomial/PolynomialTest:length}\index{length() (Java method)}

\begin{fulllineitems}
\phantomsection\label{com/linuxluigi/polynomial/PolynomialTest:com.linuxluigi.polynomial.PolynomialTest.length()}\pysiglinewithargsret{public void \sphinxbfcode{length}}{}{}
Probiert zwischen -1000 bis 1000 alle Längen durch und überprüft ob die funktion length den erwarteten Wert zurück gibt.
\begin{quote}\begin{description}
\item[{Wirft}] \leavevmode\begin{itemize}
\item {} 
\href{http://docs.oracle.com/javase/6/docs/api/java/lang/Exception.html}{\textbf{\texttt{Exception}}} -- 

\end{itemize}

\end{description}\end{quote}

\end{fulllineitems}



\subparagraph{set}
\label{com/linuxluigi/polynomial/PolynomialTest:set}\index{set() (Java method)}

\begin{fulllineitems}
\phantomsection\label{com/linuxluigi/polynomial/PolynomialTest:com.linuxluigi.polynomial.PolynomialTest.set()}\pysiglinewithargsret{public void \sphinxbfcode{set}}{}{}
Fügt in mehren Polynomen
\begin{quote}\begin{description}
\item[{Wirft}] \leavevmode\begin{itemize}
\item {} 
\href{http://docs.oracle.com/javase/6/docs/api/java/lang/Exception.html}{\textbf{\texttt{Exception}}} -- 

\end{itemize}

\end{description}\end{quote}

\end{fulllineitems}



\subsubsection{TerminalInterface}
\label{com/linuxluigi/polynomial/TerminalInterface:terminalinterface}\label{com/linuxluigi/polynomial/TerminalInterface::doc}\index{TerminalInterface (Java class)}

\begin{fulllineitems}
\phantomsection\label{com/linuxluigi/polynomial/TerminalInterface:com.linuxluigi.polynomial.TerminalInterface}\pysigline{ class \sphinxbfcode{TerminalInterface}}
User Terminal Interface Ausgabe Gibt ein Menu und sonstige nützliche Userinterface features aus Created by Steffen Exler on 18.10.16.

\end{fulllineitems}



\paragraph{Methods}
\label{com/linuxluigi/polynomial/TerminalInterface:methods}

\subparagraph{BoarderText}
\label{com/linuxluigi/polynomial/TerminalInterface:boardertext}\index{BoarderText(String) (Java method)}

\begin{fulllineitems}
\phantomsection\label{com/linuxluigi/polynomial/TerminalInterface:com.linuxluigi.polynomial.TerminalInterface.BoarderText(String)}\pysiglinewithargsret{ void \sphinxbfcode{BoarderText}}{\href{http://docs.oracle.com/javase/6/docs/api/java/lang/String.html}{String}\emph{ Text}}{}
Gibt den String Text in ein Rahm aus
\begin{quote}\begin{description}
\item[{Parameter}] \leavevmode\begin{itemize}
\item {} 
\textbf{\texttt{Text}} -- String der im Rahmen angezeigt werden soll

\end{itemize}

\end{description}\end{quote}

\end{fulllineitems}



\subparagraph{InputDouble}
\label{com/linuxluigi/polynomial/TerminalInterface:inputdouble}\index{InputDouble(String) (Java method)}

\begin{fulllineitems}
\phantomsection\label{com/linuxluigi/polynomial/TerminalInterface:com.linuxluigi.polynomial.TerminalInterface.InputDouble(String)}\pysiglinewithargsret{ double \sphinxbfcode{InputDouble}}{\href{http://docs.oracle.com/javase/6/docs/api/java/lang/String.html}{String}\emph{ TextError}}{}
Ließt eine User Terminal eingabe und überprüft ob es sich um ein double handelt und gibt diesen zurück
\begin{quote}\begin{description}
\item[{Parameter}] \leavevmode\begin{itemize}
\item {} 
\textbf{\texttt{TextError}} -- Text der bei Falscher eingabe wiederholt wird

\end{itemize}

\item[{Rückgabe}] \leavevmode
User eingabe als Double

\end{description}\end{quote}

\end{fulllineitems}



\subparagraph{InputInt}
\label{com/linuxluigi/polynomial/TerminalInterface:inputint}\index{InputInt(String) (Java method)}

\begin{fulllineitems}
\phantomsection\label{com/linuxluigi/polynomial/TerminalInterface:com.linuxluigi.polynomial.TerminalInterface.InputInt(String)}\pysiglinewithargsret{ int \sphinxbfcode{InputInt}}{\href{http://docs.oracle.com/javase/6/docs/api/java/lang/String.html}{String}\emph{ TextError}}{}
Ließt eine User Terminal eingabe und überprüft ob es sich um ein Int handelt und gibt diesen zurück
\begin{quote}\begin{description}
\item[{Parameter}] \leavevmode\begin{itemize}
\item {} 
\textbf{\texttt{TextError}} -- Text der bei Falscher eingabe wiederholt wird

\end{itemize}

\item[{Rückgabe}] \leavevmode
User eingabe als Int

\end{description}\end{quote}

\end{fulllineitems}



\subparagraph{InputString}
\label{com/linuxluigi/polynomial/TerminalInterface:inputstring}\index{InputString(String, String) (Java method)}

\begin{fulllineitems}
\phantomsection\label{com/linuxluigi/polynomial/TerminalInterface:com.linuxluigi.polynomial.TerminalInterface.InputString(String, String)}\pysiglinewithargsret{ \href{http://docs.oracle.com/javase/6/docs/api/java/lang/String.html}{String} \sphinxbfcode{InputString}}{\href{http://docs.oracle.com/javase/6/docs/api/java/lang/String.html}{String}\emph{ TextError}, \href{http://docs.oracle.com/javase/6/docs/api/java/lang/String.html}{String}\emph{ Default}}{}
Ließt eine User Terminal eingabe und ueberprueft ob es sich um ein String handelt und gibt diesen zurück
\begin{quote}\begin{description}
\item[{Parameter}] \leavevmode\begin{itemize}
\item {} 
\textbf{\texttt{TextError}} -- Text der bei Falscher eingabe wiederholt wird

\item {} 
\textbf{\texttt{Default}} -- Return Wert wenn User keine eingabe tätigt

\end{itemize}

\item[{Rückgabe}] \leavevmode
User eingabe als String

\end{description}\end{quote}

\end{fulllineitems}



\subparagraph{ShowMenu}
\label{com/linuxluigi/polynomial/TerminalInterface:showmenu}\index{ShowMenu(String{[}{]}, boolean) (Java method)}

\begin{fulllineitems}
\phantomsection\label{com/linuxluigi/polynomial/TerminalInterface:com.linuxluigi.polynomial.TerminalInterface.ShowMenu(String__, boolean)}\pysiglinewithargsret{ int \sphinxbfcode{ShowMenu}}{\href{http://docs.oracle.com/javase/6/docs/api/java/lang/String.html}{String}{[}{]}\emph{ MenuList}, boolean\emph{ Back}}{}
Erstellt ein User Terminal Menu, dieser kann mit der Int eingabe auswählen welchen Menupunkt er auswählen möchte. Das Menu wird mithilfe eines String{[}{]} gebildet und gibt die Usereingabe zurück.
\begin{quote}\begin{description}
\item[{Parameter}] \leavevmode\begin{itemize}
\item {} 
\textbf{\texttt{MenuList}} -- Eine Liste mit allen Antwortmöglichkeiten

\item {} 
\textbf{\texttt{Back}} -- True == fügt ein Menupunkt ein, um ins Vorherige Menu zurück zu kommen

\end{itemize}

\item[{Rückgabe}] \leavevmode
User Antwort als Int Wert. Der Wert ist die Nummer im MenuList{[}{]}. Beispiel: Bei MenuList{[}''Ich'', ``Du'', ``Er''{]} gibt der User 2 an und meint damit ``Du'' und 1 wird auch als Int zurück gegeben.

\end{description}\end{quote}

\end{fulllineitems}



\chapter{Sonstiges}
\label{index:sonstiges}

\section{Lizenz}
\label{license:lizenz}\label{license::doc}
\index{Lizenz}
MIT License

Copyright (c) 2016 Steffen Exler

Hiermit wird unentgeltlich jeder Person, die eine Kopie der Software und der zugehörigen Dokumentationen (die ``Software'') erhält, die Erlaubnis erteilt, sie uneingeschränkt zu nutzen, inklusive und ohne Ausnahme mit dem Recht, sie zu verwenden, zu kopieren, zu verändern, zusammenzufügen, zu veröffentlichen, zu verbreiten, zu unterlizenzieren und/oder zu verkaufen, und Personen, denen diese Software überlassen wird, diese Rechte zu verschaffen, unter den folgenden Bedingungen:

Der obige Urheberrechtsvermerk und dieser Erlaubnisvermerk sind in allen Kopien oder Teilkopien der Software beizulegen.

DIE SOFTWARE WIRD OHNE JEDE AUSDRÜCKLICHE ODER IMPLIZIERTE GARANTIE BEREITGESTELLT, EINSCHLIEßLICH DER GARANTIE ZUR BENUTZUNG FÜR DEN VORGESEHENEN ODER EINEM BESTIMMTEN ZWECK SOWIE JEGLICHER RECHTSVERLETZUNG, JEDOCH NICHT DARAUF BESCHRÄNKT. IN KEINEM FALL SIND DIE AUTOREN ODER COPYRIGHTINHABER FÜR JEGLICHEN SCHADEN ODER SONSTIGE ANSPRÜCHE HAFTBAR ZU MACHEN, OB INFOLGE DER ERFÜLLUNG EINES VERTRAGES, EINES DELIKTES ODER ANDERS IM ZUSAMMENHANG MIT DER SOFTWARE ODER SONSTIGER VERWENDUNG DER SOFTWARE ENTSTANDEN.


\section{Kontakt}
\label{license:kontakt}
\index{Kontakt}
Fragen? Kontaktieren sie \href{mailto:Steffen.Exler@gmail.com}{Steffen.Exler@gmail.com}


\section{Dokumentation}
\label{docs:dokumentation}\label{docs::doc}
\index{Sphinx}\begin{description}
\item[{Die Dokumentation ist mit \href{http://www.sphinx-doc.org/en/1.4.8/}{sphinx}, \href{https://bronto.github.io/javasphinx/}{javasphinx}}] \leavevmode
und \href{https://en.wikipedia.org/wiki/Javadoc}{Javadoc}  erstellt wordenden.

\end{description}

Gehostet wird die Dokumentation auf \href{https://readthedocs.org/}{readthedocs.org} welches durch ein Github hook mit jeden Push automatisch aktualisiert wird.
\begin{itemize}
\item {} 
\href{https://readthedocs.org/projects/polynomials-calculator/}{Online Dokumentation Link}

\item {} 
\href{https://github.com/linuxluigi/polynomials-calculator/tree/master/docs}{Github Docs Quell Datein}

\end{itemize}


\subsection{Dokumentation bearbeiten}
\label{docs:dokumentation-bearbeiten}
Die Dokumentation Quelldaten befinden sich in den Ordner \sphinxtitleref{/docs/source} und sind in reStructuredText Format geschrieben.
Nach dem bearbeiten der Quelldaten müssen diese noch in HTML konvertiert werden, dieses wird über das Shell Script
\sphinxtitleref{/docs/javasphinx.sh} erledigt.
\begin{itemize}
\item {} 
\href{http://docutils.sourceforge.net/docs/user/rst/quickref.html}{reStructuredText Schnellhilfe}

\item {} 
\href{https://www.youtube.com/watch?v=oJsUvBQyHBs}{YouTube - Sphinx \& Read the Docs}

\item {} 
\href{http://socrates.io/}{socrates.io - reStructuredText WYSIWYG Editor}

\end{itemize}


\subsection{Dokumentation aktualisieren}
\label{docs:dokumentation-aktualisieren}
Es wurde für Ubuntu 12.04, 14.04 und 16.04 mit Python 3 ein Shell Script zur automatischen konvertierung von Javadoc und reStructuredText Datein
zur HTML integrierd, auf welches \href{https://readthedocs.org/}{readthedocs.org} zugreift sobald ein push auf Github gesendet wird.


\subsubsection{Abhänigkeiten installieren}
\label{docs:abhanigkeiten-installieren}
\begin{Verbatim}[commandchars=\\\{\}]
\PYGZdl{} sudo apt\PYGZhy{}get build\PYGZhy{}dep python\PYGZhy{}lxml
\PYGZdl{} sudo apt\PYGZhy{}get install texlive\PYGZhy{}full
\end{Verbatim}

Nur für Ubuntu 12.04 und 14.04

\begin{Verbatim}[commandchars=\\\{\}]
\PYGZdl{} sudo apt\PYGZhy{}get install python\PYGZhy{}virtualenv
\end{Verbatim}

Für Ubuntu 16.04

\begin{Verbatim}[commandchars=\\\{\}]
\PYGZdl{} sudo apt\PYGZhy{}get install python3\PYGZhy{}venv
\end{Verbatim}


\subsubsection{Virtualenv anlegen und verwenden}
\label{docs:virtualenv-anlegen-und-verwenden}
\textbf{wichtig} \textgreater{}\textgreater{} folgene 2 Befehle im Wurzelverzeichnis des Projektes ausführen!

Virtualenv für Python 3 anlegen

\begin{Verbatim}[commandchars=\\\{\}]
\PYGZdl{} virtualenv \PYGZhy{}p python3 env
\end{Verbatim}

In virtuelle Umgebung einloggen

\begin{Verbatim}[commandchars=\\\{\}]
\PYGZdl{} \PYG{n+nb}{source} env/bin/activate
\end{Verbatim}


\subsubsection{Python abhänigkeiten installieren}
\label{docs:python-abhanigkeiten-installieren}
\begin{Verbatim}[commandchars=\\\{\}]
\PYGZdl{} pip install \PYGZhy{}r docs/requirements.txt
\end{Verbatim}


\subsubsection{Dokumentation erzeugen}
\label{docs:dokumentation-erzeugen}
Im Unterverzeichnis /docs wechseln und das Script javaspinx.sh ausführen

\begin{Verbatim}[commandchars=\\\{\}]
\PYGZdl{} ./javasphinx.sh
\end{Verbatim}

Sobald das Script erfolgreich ausgeführt wurde sind in den Order \sphinxtitleref{/docs/build/} die Aktuelle Dokumentation in verschiedenen Formaten zu finden.


\subsubsection{Dokumentation alternative Formate}
\label{docs:dokumentation-alternative-formate}
Es ist über die \sphinxtitleref{Makefile} in \sphinxtitleref{/docs} wird die Dokumentation in mehreren Formaten ausgegeben:
\begin{itemize}
\item {} 
EPUB --\textgreater{} /docs/build/epub/PolynomialsCalculator.epub

\item {} 
epub3 --\textgreater{} /docs/build/epub3/PolynomialsCalculator.epub

\item {} 
latex --\textgreater{} /docs/build/latex/

\item {} 
PDF --\textgreater{} /docs/build/latex/PolynomialsCalculator.pdf

\item {} 
man --\textgreater{} /docs/build/man/polynomialscalculator.1

\end{itemize}


\section{Hilfe}
\label{help:hilfe}\label{help::doc}
\index{Support}
Wenn Sie hilfe brauchen email \href{mailto:Steffen.Exler@gmail.com}{Steffen.Exler@gmail.com}


\chapter{Indices and tables}
\label{index:indices-and-tables}\begin{itemize}
\item {} 
\DUrole{xref,std,std-ref}{genindex}

\item {} 
\DUrole{xref,std,std-ref}{modindex}

\item {} 
\DUrole{xref,std,std-ref}{search}

\end{itemize}



\renewcommand{\indexname}{Stichwortverzeichnis}
\printindex
\end{document}
